\nocite{*}
\section{Introduction}
Interviews are often hard to be judged. It is often left in the hands of the interviewer(s) to measure the hirability of the candidates. This is fundamentally flawed as this heavily relies of interviewers' mood and personality. Also, in most cases multiple interviewers interview for the same roles which makes this process even less scientific as it is almost impossible to fairly aggregate the opinions of interviewers.

There has been tons of research by psychologists and career experts about what one should do in order to succeed in an interview \cite{huffcutt2001identification}. From this, we know that things like smiling, using a confident tone and making good eye contact can contribute a lot in an interview. However, these observations are often based on intuition and experience. Hence, It is hard to automate and quantify hirability of candidates. Also, there is a common misconception that content of the interviewee's responses is the sole determinant of the job interview. However, it is seen that non verbal aspects are as important if not more important than verbal responses \cite{mehrabian1971silent}.

In this project we would like to build a computational framework using which interviewers and interviewees can use it to analyze interviews and obtain the following.
\begin{itemize}
\item Automatically predict the overall score of the interview.
\item Quantify verbal and nonverbal behavior of the interviewee towards the success in the interview.
\item Automatically recommend aspects to be improved for better overall score. 
\item Timeline that shows how well the interview progressed with respect to time.
\end{itemize}

In order to achieve this we propose a framework as shown in the figure below. We use a one on one interview data comprised of three modes (audio, video and textual). Then, we extract multimodal features (facial expressions, lexical and prosody) and predict the overall score of the interview, how likely the candidate is going to be hired and other traits required for the interview process.